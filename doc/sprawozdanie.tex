%%%%%%%%%%%%%%%%%%%%%%%%%%%%%%%%%%%%%%%%%
% University/School Laboratory Report
% LaTeX Template
% Version 3.1 (25/3/14)
%
% This template has been downloaded from:
% http://www.LaTeXTemplates.com
%
% Original author:
% Linux and Unix Users Group at Virginia Tech Wiki 
% (https://vtluug.org/wiki/Example_LaTeX_chem_lab_report)
%
% License:
% CC BY-NC-SA 3.0 (http://creativecommons.org/licenses/by-nc-sa/3.0/)
%
%%%%%%%%%%%%%%%%%%%%%%%%%%%%%%%%%%%%%%%%%

%----------------------------------------------------------------------------------------
%	PACKAGES AND DOCUMENT CONFIGURATIONS
%----------------------------------------------------------------------------------------

\documentclass{article}

\usepackage{siunitx} % Provides the \SI{}{} and \si{} command for typesetting SI units
\usepackage{graphicx} % Required for the inclusion of images
%\usepackage{natbib} % Required to change bibliography style to APA
\usepackage{amsmath} % Required for some math elements 
\usepackage[margin=1.5in]{geometry}


\setlength\parindent{0pt} % Removes all indentation from paragraphs

\renewcommand{\labelenumi}{\alph{enumi}.} % Make numbering in the enumerate environment by letter rather than number (e.g. section 6)

%ustawienie jezyka polskiego
\usepackage{polski}
\usepackage[utf8]{inputenc}
\usepackage[T1]{fontenc}
\usepackage{multirow}
\usepackage{rotating}
\usepackage{float}
\usepackage[table]{xcolor}
\usepackage{enumerate}
\usepackage{subcaption}


\graphicspath{ {images/} }


%----------------------------------------------------------------------------------------
%	DOCUMENT INFORMATION
%----------------------------------------------------------------------------------------

\title{Inteligencja Obliczeniowa\\
	\vspace{5mm}
	\textbf{Laboratorium 3-5}
}

\author{\\
	\\\textbf{Autorzy:}
	\\Maciej Kiedrowski, nr indeksu: 200105
	\\Wojciech Then, nr indeksu: 
	\\\\
	\\
	\\\textbf{Grupa:} Środa 18:45}
\date{\textbf{Data oddania:} 06.05.2017}

\begin{document}
	
	\maketitle % Insert the title, author and date
	
	\begin{center}
		\begin{tabular}{l r}
			%Data wykonania ćwiczenia: & ... \\ % Date the experiment was performed
			%Partners: & James Smith \\ % Partner names
			%& Mary Smith \\
			Prowadzący: & Dr hab. inż. Olgierd Unold 
			
		\end{tabular}
	\end{center}
	
	
	\newpage
	\tableofcontents 	%spis tresci
	\newpage
	
	\section{Własne funkcje krzyżowania i mutacji}

Testy algorytmu genetycznego dla własnych funkcji krzyżowania oraz mutacji

Zmiana prawdopodobieństwa krzyżowania

\begin{figure}[H]
	\centering
	\includegraphics[scale=0.5]{custCrossover_pcross_result}
	\caption{Rezultat optymalizacji}
	\label{rys:fftFiltry}
\end{figure}

\begin{figure}[H]
	\centering
	\includegraphics[scale=0.5]{custCrossover_pcross_iterations}
	\caption{Rezultat optymalizacji}
	\label{rys:fftFiltry}
\end{figure}


Zmiana ilości maksymalnej ilości iteracji

\begin{figure}[H]
	\centering
	\includegraphics[scale=0.5]{custCrossover_maxIter_result}
	\caption{Rezultat optymalizacji}
	\label{rys:fftFiltry}
\end{figure}

\begin{figure}[H]
	\centering
	\includegraphics[scale=0.5]{custCrossover_maxIter_iterations}
	\caption{Rezultat optymalizacji}
	\label{rys:fftFiltry}
\end{figure}
%	\newpage
%	\input{./chapters/Zalozenia}
%	\newpage
%	\input{./chapters/Implementacja}
%	\newpage
%	\input{./chapters/Proby}
%	\newpage
%	\input{./chapters/Wnioski}

	% If you wish to include an abstract, uncomment the lines below
	% \begin{abstract}
	% Abstract text
	% \end{abstract}
	
	%----------------------------------------------------------------------------
	
	
	%\begin{figure}[h]
	%	\begin{center}
	%		\includegraphics[width=0.65\textwidth]{placeholder} % Include the image placeholder.png
	%		\caption{Figure caption.}
	%	\end
	%	{center}
	%\end{figure}
	
	
\end{document}